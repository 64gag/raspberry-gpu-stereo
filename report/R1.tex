\documentclass[paper=a4, fontsize=11pt]{scrartcl} % A4 paper and 11pt font size

\usepackage[T1]{fontenc} 
\usepackage{fourier} 
\usepackage[english]{babel}
\usepackage{amsmath,amsfonts,amsthm} 

\usepackage{lipsum}

\usepackage{sectsty}
\allsectionsfont{\centering \normalfont\scshape} 

\usepackage{fancyhdr} 
\pagestyle{fancyplain}
\fancyhead{} 
\fancyfoot[L]{} 
\fancyfoot[C]{}
\fancyfoot[R]{\thepage}
\renewcommand{\headrulewidth}{0pt}
\renewcommand{\footrulewidth}{0pt}
\setlength{\headheight}{13.6pt}

\numberwithin{equation}{section}
\numberwithin{figure}{section}
\numberwithin{table}{section}

\setlength\parindent{0pt}

\newcommand{\horrule}[1]{\rule{\linewidth}{#1}} 

\title{	
\normalfont \normalsize 
\textsc{Universidad de Monterrey} \\ [25pt] 
\horrule{1pt} \\[0.4cm]
\huge Reporte de actividades \\ 
\small Semanas 1 y 2 \\
\horrule{2pt} \\[0.5cm]
}

\author{Pedro Aguiar}

\date{\normalsize 2/Mayo/2014}

\begin{document}

\maketitle

Mi primer d\'{i}a en LCIS fue el mi\'{e}rcoles 23 de Abril. La primer semana solamente hubo tres d\'{i}as laborales para m\'{i}. En esta semana todav\'{i}a hubo pendientes administrativos que se reportan a continuaci\'{o}n, como referencia e informaci\'{o}n para futuros estudiantes de la UDEM que vayan a venir a un laboratorio similar.

\section{Procesos administrativos}
Al llegar al laboratorio lo primero que hiceron fue llevarme con Jennyfer DUBERVILLE para revisar los procedimientos administrativos. Se revis\'{o} que la papeler\'{i}a que hab\'{i}a sido enviada por internet estuviera correcta. La papeler\'{i}a consisti\'{o} en:
\begin{itemize}
	\item Credencial de estudiante
	\item Copia del \'{u}ltimo certificado (en mi caso el de la preparatoria)
	\item Copia del pasaporte
	\item Un contrato (firmado por m\'{i} y por el director de DIT, por parte de la UDEM)
\end{itemize}
Despu\'{e}s de corroborar la papeler\'{i}a se le entreg\'{o} un formulario a mi asesor y a m\'{i} me dieron un reglamento acerca del uso de los recursos del laboratorio (internet, equipo electr\'{o}nico, etc\'{e}tera). Se me hizo firmar un papel en el que dec\'{i}a la fecha de llegada al laboratorio y donde declaraba haber le\'{i}do el reglamento. Se me record\'{o} que los primeros d\'{i}as de Mayo tendr\'{i}a que pagar las cuotas de "social security" y "civil liability". Tambi\'{e}n se me inform\'{o} que pare recibir la gratificaci\'{o}n tengo que abrir una cuenta en un banco Franc\'{e}s y me dijeron que me iban a contactar con una persona que pod\'{i}a ayudarme con eso.

Por \'{u}ltimo se me proporcionaron las credenciales de acceso a Internet y mi computadora. Se instal\'{o} el equipo de c\'{o}mputo que se me asign\'{o} y se me explic\'{o} c\'{o}mo utilizarlo, as\'{i} como algunos comentarios finales (no se puede conectar nada a la red al\'{a}mbrica, no puedo instalar software que no est\'{e} en los repositorios de Debian, c\'{o}mo acceder a mi espacio en el servidor de la escuela que est\'{a} respaldado, detalles de ese tipo).

\section{Trabajo}
\subsection{PCB comunicaci\'{o}n entre c\'{a}mara y raspberry pi}
Durante el primer d\'{i}a de trabajo se hizo un bill of materials para hacer una tarjeta de circuito impreso que conecte una c\'{a}mara OV7670 con el raspberry, el objetivo siendo que el Carrita tenga capacidades de visi\'{o}n computarizada. La diferencia con la de la UDEM es que Varrier extern\'{o} que le gustar\'{i}a recibir en el raspberry:
\begin{itemize}
	\item M\'{a}xima resoluci\'{o}n (640x480).
	\item M\'{a}ximo framerate (30fps).
	\item Imagen a color.
\end{itemize}
El bill of materials generado est\'{a} en la siguiente liga: http://goo.gl/PQ6T6W . Y de la lista se pueden hacer las siguientes observaciones:
\begin{itemize}
	\item Costo elevado
	\begin{itemize}
		\item 40.74 euros de componentes de una placa.
		\item 33.99 euros para el programador.
	\end{itemize}
	\item El problema es que el proveedor impone cantidades m\'{i}nimas.
	\item No est\'{a} considerado el costo de mandar a hacer la placa.
\end{itemize}
Adem\'{a}s hay que observar que:
\begin{itemize}
	\item No se podr\'{i}a utilizar el microcontrolador de la UDEM porque no estaban considerados los bits para el color.
	\item Por lo tanto el PCB tendr\'{i}a que ser diseñado completamente desde cero con la sensaci\'{o}n (personal) de menos herramientas y proveedores menos accesibles.
\end{itemize}

\subsection{Alternativa a Microchip}
A pesar de haber entregado un bill of materials proced\'{i} a buscar una alternativa que ofrecerle a Varrier, porque me parec\'{i}a que la soluci\'{o}n era muy costosa en tiempo y dinero, y nada conveniente para nadie. La mejor alternativa fue utilizar un STM32, porque:
\begin{itemize}
	\item Tienen un perif\'{e}rico para c\'{a}maras digitales (DCMI) y DMA.
	\begin{itemize}
		\item El DCMI se encarga de procesar las im\'{a}genes de la c\'{a}mara.
		\item Junto con el DMA se puede guardar todo en memoria usando s\'{o}lo hardware (el procesador queda libre para usarlo para otras aplicaciones).
	\end{itemize}
	\item Hay tarjetas de evaluaci\'{o}n de bajo costo (una de 12 y otra de 20 euros).
	\item Programaci\'{o}n por USB.
	\item Varrier tiene experiencia y confianza en estos dispositivos.
\end{itemize}
Le pareci\'{o} bien la propuesta de utilizar estos dispositivos as\'{i} que proced\'{i} a validar que las tarjetas de evaluaci\'{o}n pudieran hacer la comunicaci\'{o}n con la c\'{a}mara. Tambi\'{e}n se revis\'{o} si se podr\'{i}an utilizar los sensores y la pantalla LCD de la tarjeta (si no interfer\'{i}an con los pines que se necesitaban para la c\'{a}mara). Se tom\'{o} la decisi\'{o}n de comprar dos tarjetas como las de la siguiente liga: http://goo.gl/43VuOs .

\subsection{Visi\'{o}n computarizada}
Hechos:
\begin{itemize}
	\item Es un objetivo que el Carrita pueda medir la distancia que hay hacia un veh\'{i}culo que est\'{e} enfrente (para implementar control de distancia).
	\item Esto no est\'{a} implementado en el c\'{o}digo de la UDEM.
	\item El c\'{o}digo de la UDEM usa OpenCV
\end{itemize}

Se decidi\'{o} deslindarse de OpenCV y desarrollar c\'{o}digo en C para una transformada de Hough basado en los siguientes hechos:

\begin{itemize}
	\item Para la detecci\'{o}n de l\'{i}nea s\'{o}lo se estaban usando dos "funciones complejas" de OpenCV:
	\begin{itemize}
		\item Canny edge detector ( http://goo.gl/WSTXvf )
		\item Hough transform ( http://goo.gl/ZFBRKU )
	\end{itemize}
	\item La transformada de Hough puede usarse para detectar al veh\'{i}culo de enfrente, pero no la de OpenCV.
	\item  Las ventajas t\'{i}picas de programar una versi\'{o}n independiente de librer\'{i}as:
	\begin{itemize}
		\item Minimalista, r\'{a}pida.
		\item Conocimiento profundo de la transformada.
		\item Libertad en funcionalidad.
		\item Se puede hacer "a la medida" del raspberry (aunque con OpenCV se pueden hacer muchas cosas y hay videos al respecto, en alg\'{u}n momento para empujar ese l\'{i}mite se tendr\'{i}a que bajar a soluciones espec\'{i}ficas).
	\end{itemize}
\end{itemize}

\subsubsection{Hough-based tracking}
Adem\'{a}s de las razones mencionadas anteriormente, la m\'{a}s determinante es que la funci\'{o}n de OpenCV no considera el frame anterior para identificar objetos. Hace el 100\% del c\'{a}lculo en cada frame. Y es algo de lo que se puede tomar ventaja instant\'{a}neamente: utilizar la ubicaci\'{o}n previa de la l\'{i}nea o el veh\'{i}culo de enfrente para empezar a buscar alrededor de ese punto. Se est\'{a} haciendo una revisi\'{o}n bibliogr\'{a}fica del tema y se reportar\'{a}n los resultados la pr\'{o}xima semana. De momento parece que es algo nuevo en raspberries y es un área que me motiva mucho.

\subsubsection{Estado actual del desarrollo}
Ya est\'{a} listo lo mismo que hac\'{i}a el c\'{o}digo de OpenCV, pero sin usar OpenCV. Desde que entra la imagen hasta que sale con las l\'{i}neas dibujadas encima. Para la siguiente semana reportar\'{e} benchmarkings e implementar\'{e} funcionalidad nueva, dependiendo del avance en la revisi\'{o}n bibliogr\'{a}fica.

\section{Siguiente semana}
\begin{itemize}
\item Compras: Varrier me estim\'{o} que la pr\'{o}xima semana estar\'{a}n llegando los componentes.
\item Repositorio y Dropbox: Estuve buscando una forma de sincronizar y respaldar el trabajo, ya me rend\'{i} y parece que no habr\'{a} formas c\'{o}modas (por restricciones de no instalar software que no est\'{e} en los repositorios de Debian y debido a que estamos atr\'{a}s de un proxy que necesita contraseñas), pero aun as\'{i} subir\'{e} todo el trabajo a alg\'{u}n lugar, le aviso la decisi\'{o}n en el pr\'{o}ximo reporte.
\item Gantt: Le puedo adelantar que Varrier y yo estuvimos de acuerdo en que en dos meses tenemos que haber hecho el control de distancia (y de ser posible tambi\'{e}n lateral) utilizando las c\'{a}maras y haber env\'{i}ado un art\'{i}culo a alguna conferencia. Varrier s\'{o}lo vino el mi\'{e}rcoles de esta semana (ten\'{i}a compromisos fuera) y yo obviamente no ten\'{i}a listo el Gantt para propon\'{e}rselo (ese d\'{i}a hablamos esto). Si el lunes viene y lo aprueba se lo env\'{i}o el mismo lunes. Despu\'{e}s de esos dos meses me dijo que lo decidiriamos despu\'{e}s en base a mis intereses, etc\'{e}ter\'{\'{a}}. La \'{u}nica opci\'{o}n que le ofrec\'{i} fue la de simular la flotilla de veh\'{i}culos y le pareci\'{o} buena idea pero me dijo que lo siguiera pensando y cuando estuviera la fecha m\'{a}s cercana tom\'{a}bamos la decisi\'{o}n. Entiendo algo como que el plan de colaborar con otro equipo (que se iba a encargar de la simulaci\'{o}n de la flotilla) se vino abajo.
\item Las actividades se dibujan as\'{i}: programaci\'{o}n de STM32, trabajo electr\'{o}nico para env\'{i}ar las im\'{a}genes de la c\'{a}mara al raspberry, desarrollo de transformada de hough espec\'{i}fica para la aplicaci\'{o}n, sacar modelos adecuados para la aplicaci\'{o}n, implementar control tolerante a fallas.
\end{itemize}
\end{document}