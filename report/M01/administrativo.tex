\subsection{Proceso Administrativo}
En esta semana todavía hubo pendientes administrativos que se reportan a continuación, como referencia e información para futuros estudiantes de la UDEM que vayan a participar en un programa similar. Al llegar al laboratorio lo primero que hiceron fue llevarme con Jennyfer DUBERVILLE para revisar los procedimientos administrativos. Se revisó que la papelería que había sido enviada por internet estuviera correcta. La papelería consistió en:
\begin{itemize}
	\item Credencial de estudiante
	\item Copia del último certificado (en mi caso el de la preparatoria)
	\item Copia del pasaporte
	\item Un contrato (firmado por mí y por el director de DIT, por parte de la UDEM)
\end{itemize}
Después de corroborar la papelería se le entregó un formulario a mi asesor y a mí me dieron un reglamento acerca del uso de los recursos del laboratorio (internet, equipo electrónico, etcétera). Se me hizo firmar un papel en el que decía la fecha de llegada al laboratorio y donde declaraba haber leído el reglamento. Se me recordó que los primeros días de Mayo tendría que pagar las cuotas de "social security" y "civil liability". También se me informó que pare recibir la gratificación tengo que abrir una cuenta en un banco Francés y me dijeron que me iban a contactar con una persona que podía ayudarme con eso.

Por último se me proporcionaron las credenciales de acceso a Internet y mi computadora. Se instaló el equipo de cómputo que se me asignó y se me explicó cómo utilizarlo, así como algunos comentarios finales (no se puede conectar nada a la red alámbrica, no puedo instalar software que no esté en los repositorios de Debian, cómo acceder a mi espacio en el servidor de la escuela que está respaldado, detalles de ese tipo).

